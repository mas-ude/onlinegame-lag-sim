%!TEX root = paper.tex
%%%%%%%%%%%%%%%%%%%%%%%%%%%%%%%%%%%%%%%%%%%%%%%%%%%%%%%%%%%%%%%%%%%%%%%%%%%%%%%%
\section{Choosing Representative Games}
\label{sec:game-criteria}

Before proposing actual measurement methods, three equally important aspects relating to any game measurement approch are discussed: the metrics to evaluate, the parameters to use, as well as the selection of games. 

Finding fitting objective \gls{QoE} metrics is strongly game-dependent. They need to reflect the game's mechanics in a meaningful manner. But this is beyond the scope of this paper, which will only concern itself with game-independent metrics, of which three are considered here: The framerate, the frame \gls{IAT}, and the end-to-end lag (or parts thereof as some measurement methods might not be able to record the full lag).

The framerate and \gls{IAT} are a good representation for the game's fluidity and responsiveness to inputs and are especially important to note for cloud gaming. The end-to-end lag gives the best overall picture on the attainable gaming experience and should always be preferred over partial lag values, such as purely investigating the network latency. The impact of the lag also depends on the type and precision of controls the game offers. A keyboard and mouse driven PC game might be much more sensitive to high lag than a mobile game with touch controls.

Input controls are but one aspect of the games environment and settings, which also need to be carefully selected to achieve meaningful results. This especially concerns PC games which usually offer a wide range of options to choose from of which the graphics options are the most impactful. The recommended settings to run games at are a video resolution of 1080p or higher with the games other graphics options set to high or at lest medium values in order to reflect a typical gaming experience. The test computers should be able to deliver these graphical settings at a framerate of \SI{60}{\hertz} which is considered the minimum rate for many players. Some types of games are less dependent on the framerate, where a rate of \SI{30}{\hertz} is considered acceptable. Experimenters should never set a framerate lower than that for the reasons discussed in the previous section. They should however also consider testing at higher framerates, especially for competitive games with a high tickrate to further reduce the negative impact of low framerates.

%The general focus here lies on measuring online games with high demands. The idea is that if these games work at an objectively good quality, it is reasonable to assume that all other games do so as well. 
It is quite difficult to find games that are representative for certain input and lag demands. For example, the traditional game genre categorization is not a good starting point as games from the same category can be vastly different in terms of game speed and necessary reaction times. Rather, the following four metrics might be more helpful to assess a game's applicability for measurements.

 \begin{itemize}
    \item \textbf{Required number of decisions or actions in a certain time span.} E.g., the \gls{CCG} \textsc{Hearthstone} may only require a handful of actions (i.e., choosing and playing cards) each turn, while in order to competitively play the \gls{RTS} \textsc{Starcraft 2} you more or less need to achieve a few hundred so-called \gls{APM}, with the record being higher than $800$ \gls{APM}.% \cite{6404025} defines a related metric dubbed \textit{command heaviness} comparing the amount of change to the input rate, under the umbrella term \textit{real-time strictness}. This ties in with the concepts of game sense. Micromanagement-intense games usually tend to result in high APM rates.

    \item \textbf{Maximum successful reaction time to in-game actions.} Again e.g., \textsc{Hearthstone} as a turn-based game requires no instant reaction time at all, as the opponent's and the player's actions are separated into turns. First-person shooters like \textsc{Counter-Strike: Global Offensive} are usually on the opposite extreme of this spectrum, as they tend to have a very high tickrate and literally often require you to ``shoot first'' to win. This is also investigated/defined in \cite{Claypool:2006:LPA:1167838.1167860}. This metric is also closely related to the next item.

    \item \textbf{(Un-)Predictability of actions.} Compare, e.g., an entirely rhythm-based game (e.g., \textsc{Guitar Hero}), where you can completely pre-plan all of your actions, to a twitch-based shooter like \textsc{Counter-Strike} where you have to react from moment to moment. In theory, a game with no surprising events will be much less influenced by a higher end-to-end lag.

    \item \textbf{Accuracy and precision of input actions.} Accuracy can be both in terms of temporal as well as spatial aspects which can be influenced by both the image quality and the frame rate.  %Commands through keybindings usually require less precision than freeform mouse inputs on certain regions on the screen.

\end{itemize}

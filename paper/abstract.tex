%!TEX root = paper.tex
%%%%%%%%%%%%%%%%%%%%%%%%%%%%%%%%%%%%%%%%%%%%%%%%%%%%%%%%%%%%%%%%%%%%%%%%%%%%%%%%
\begin{abstract}
The online interactions and resulting user experience of the online 
interaction of video games have been a topic of several publications in 
the past. This trend was reinforced through the development of Cloud 
Gaming with its strong dependence on good network conditions. Many of 
these studies focus on subjective user assessments, as this is usually 
an approach that quickly results in usable scores without going too 
deep into technical details of video games. Contrary, objective and 
\acrshort{QoS}-centric measurements are often much harder to conduct.

This paper aims to offer some insights and best practices for the 
realization of such objective online video game measurements. To 
facilitate this, we first discuss two properties intrinsic to video 
games which distinguish them from pure video streaming:
\begin{enumerate*} 
	\item the framerate and 
	\item the game tickrate 
\end{enumerate*}
and their impact on the game's interactivity and end-to-end lag. 
Derived from these properties three distinct measurement approaches and 
their limitations are presented.

\end{abstract}

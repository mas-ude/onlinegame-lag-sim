%!TEX root = paper.tex
%%%%%%%%%%%%%%%%%%%%%%%%%%%%%%%%%%%%%%%%%%%%%%%%%%%%%%%%%%%%%%%%%%%%%%%%%%%%%%%%
\begin{abstract}

Recently, video games have been shifting more and more into focus of 
both the public and research's interest. Online and cloud gaming might 
be the next-big-thing after HTTP adaptive video streaming for the 
fields of computer network performance evaluation and \acrshort{QoE} 
modeling it. A quickly growing number of publications concerns itself 
with assessing the subjective quality of those two game types. However, 
due to the nature of video games and their large diversity such results 
are always specific to a single game, and arguably even specific to the 
exact conditions in the experiment, and absolutely cannot be 
transferred to any other game, even if they seem to be similar on the 
surface.

This paper aims to rectify this by providing a model for an underlying 
mechanic that is a governing property of most of such earlier results: 
The end-to-end lag of video games. The paper describes in detail the 
components of this lag model and additionally provides a 
parametrizable simulation that calculates the lag.


% The online interactions and resulting user experience of the online 
%interaction of video games have been a topic of several publications in 
%the past. This trend was reinforced through the development of Cloud 
%Gaming with its strong dependence on good network conditions. Many of 
%these studies focus on subjective user assessments, as this is usually 
%an approach that quickly results in usable scores without going too 
%deep into technical details of video games. Contrary, objective and 
%\acrshort{QoS}-centric measurements are often much harder to conduct.

% This paper aims to offer some insights and best practices for the 
%realization of such objective online video game measurements. To 
%facilitate this, we first discuss two properties intrinsic to video 
%games which distinguish them from pure video streaming:
% \begin{enumerate*} 
% 	\item the framerate and 
% 	\item the game tickrate 
% \end{enumerate*}
% and their impact on the game's interactivity and end-to-end lag. 
%Derived from these properties three distinct measurement approaches and 
%their limitations are presented.

\end{abstract}

%!TEX root = paper.tex
%%%%%%%%%%%%%%%%%%%%%%%%%%%%%%%%%%%%%%%%%%%%%%%%%%%%%%%%%%%%%%%%%%%%%%%%%%%%%%%%
\section{Background}
\label{sec:background}

%Before looking at the fallacies in literature, first some fundamental gaming concepts need to be introduced. 
%Games are feedback-directed real-time simulators. 
Any game continuously reads player input, updates the game state, and renders new screen contents in a loop. In networked games, the \textit{tickrate} governs the frequency of such server-side game state updates, and the \textit{framerate} determines the client-side update rate of the output image. Popular examples for game tickrates include \SI{64}{\hertz} or \SI{128}{\hertz} for \textsc{CS:GO}, \SI{20}{\hertz} for \textsc{Minecraft}, or \SI{30}{\hertz} for \textsc{Dota 2}.

Motion in video data is based on the principle of \textit{apparent motion}, requiring a minimum framerate of about \SI{16.67}{\hertz} for motion perception to work correctly. Video media, playing at framerates of \SIrange{24}{30}{\hertz}, are considered to be at the low end of motion perception. However, stuttering can partially be concealed through motion blur. Video games have to target higher framerates: usually \SIrange{30}{144}{\hertz}, depending on the type of game. Higher framerates are especially important for increasing the interactivity and reactivity as video games constantly require input on short time scales to which the game reacts and displays the feedback.
%This circumstance is grossly underrepresented in many gaming \gls{QoE} studies by using framerates unsuitable for gaming, e.g. \SI{3}{\hertz} in \cite{claypool2007}.

Lag is a crucial factor for almost all games, as it governs the reaction time to in-game events. In literature, lag is often described solely on the basis of the network delay in an online game, neglecting other components that contribute to the lag, including the input device, the time to sample and process the input, the game engine and server and their tickrates, frame rendering time, and ultimately the time to display the frame on the monitor. The \textit{\gls{E2E} lag} can only be fully captured if all those sources are factored in. Partial approaches, like software recording of a video game might be the simplest approach to determine video game lag and framerate (e.g., cf. \cite{Chen:2011:MLC:2072298.2071991}). 
%However, this does not represent the complete \gls{E2E} lag, as both the controller and screen output delay are missing. 
But only external recording methods can fully capture the \gls{E2E} lag by simultaneously recording both the screen and input devices (e.g. \cite{beyermethod}) and noting the time between action and reaction.
%When, e.g., using a video camera, the experimenter then counts the frames between pressing a button on the input device and the action appearing on the screen and calculates the lag from this.
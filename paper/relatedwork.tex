%!TEX root = paper.tex
%%%%%%%%%%%%%%%%%%%%%%%%%%%%%%%%%%%%%%%%%%%%%%%%%%%%%%%%%%%%%%%%%%%%%%%%%%%%%%%%
\section{Related Work}
\label{sec:relatedwork}


The assessment of video games has been a topic in many past papers. Due to the strong interactivity of games and the large number of different game mechanics most of the research is focused on conducting user studies and noting the subjective quality of the users. The outcome of these studies depends strongly on a wide selection of factors (e.g., on the precise setup, the game, and the choice of players) which makes comparing their results quite difficult. This section presents a few examples of such studies.

In \cite{5976180} Jarschel et al. identify some influence factors on the subjective quality of cloud gaming through a user survey for certain games and three different game categories (slow, medium, fast games) that have been subjected to worsening \gls{QoS} parameters. Downstream packet loss and delay was noted be be especially problematic for achieving a good quality. Similarly, the authors of \cite{4591393} observed the relationship of players quitting a \gls{MMO} game with deteriorating \gls{QoS} and noted an impact proportion of 1:2:4:3 for delay, jitter, and packet loss on both the client-to-server and server-to-client connections respectively. Additionally, a user study in \cite{4604397} also showed a correlation of the \gls{QoE} to the delay as well as the jitter for another \gls{MMO}, in this case the total delay had more impact than the delay variation. Regarding the subjective quality in first person shooters, the authors of \cite{6614351} find a strong impact of the delay and packet loss on the experienced quality. The authors of \cite{6404025} and \cite{beyerusing} use \gls{fEMG} and \gls{EEG} approaches respectively to examine individual gamers' reaction to various cloud games and measure the quality they are experiencing in terms of real-time strictness and \gls{QoE}. Finally, an ITU-T Recommendation \cite{mollertowards} concerning subjectively measuring video game \gls{QoE} is also in preparation, which discusses game-relevant \gls{QoS}-metrics as well as the selection of players and games.

In order to avoid some of the issues with subjective user studies, other approaches examine the player objective performance through in-game metrics such as the game's highscore or the duration to achieve a certain task. For example, a user study in \cite{Chen:2006:SOG:1167838.1167859} observes a decrease in the objective quality assessment metrics (the playing duration) in an \gls{MMO} due to the influence of network factors. A 2006 paper \cite{Claypool:2006:LPA:1167838.1167860} categorizes player actions and their relationship to latency, with special regards for the precision and deadlines of actions. However, the in-game metrics under study can not represent short-term effects of latency as they operate on much larger time-scales. 
%(e.g. researching the whole technology tree in Warcraft 3 is neither a representative action of the game nor delay-sensitive at all). Also the presence of lag compensation techniques is not considered. The end result is a delay tolerance table which probably is not very representative of modern video games.
A second paper by Claypool et al. \cite{claypool2007} furthers this notion of the influence of network \gls{QoS} on in-game actions and specifically looks at player performance in first person games. A user study records the performance in artificially created in-game environments. Here, the performance gets worse with a degraded network, albeit at a very slow pace. %Players were also subjected to very low framerates of \SI{7}{\hertz} and \SI{3}{\hertz}, which is such an unrealistic setting, that it should not even have been tested. The reasons for this are laid out in Section~\ref{sec:framerate}.
The ``kills per minute'' of normal players in the \gls{FPS} \textsc{Quake 3} are investigated by \cite{1266180}, which sees a steady decline of this subjective performance metric when increasing the network delay. However, another paper \cite{Beigbeder:2004:ELL:1016540.1016556}, looking at player performance in \textsc{Unreal Tournament 2003} in a controlled in-game environment, finds almost no influence of increased delay and packet loss, even at high values of \SI{200}{\milli\second}. %The discrepancy between the otherwise similar studies could potentially be attributed to the specific construction of the in-game test environment. 
A 2002 paper \cite{Pantel:2002:IDR:507670.507674}, which gathered objective \gls{QoE} measurements from a custom-made RC racing video game, again sees a strong dependence of the player's performance on the delay, and suggests that a network \acrshort{RTT} of \SI{200}{\milli\second} is barely usable and \SI{500}{\milli\second} completely unusable. Finally, the authors of \cite{Bredel:2010:MSR:1944796.1944797} also find a strong and negative influence of high delay on the player's performance, in this case again in \textsc{Quake 3}. 

It is interesting to note the discrepancies between the studies both in terms of the influencing parameters as well as their degree. Considering the complexity and variability of video games and the difficulties of finding and setting up good comparable scenarios and testbeds, it is understandable that some studies come to different results. This makes it much more important to understand the basic components and underlying objective metrics better. Quality estimations of video games could be much more representative if models for these metrics would exist to easily understand their influences without having to conduct a full user study.





%Pure \gls{QoS} views on the quality of online video games are rather are, but are nonetheless important in correctly assessing any game-related parameters and are also the foundation for most objective and subjective \gls{QoE} studies.



%A 2012 article notes the dependence of gaming on latency and the difficulties cloud-based solutions have in providing a sufficiently low latency. The suggest a move closer to the edge to reach more users in a quality they deem adequate.  \cite{Choy:2012:BSC:2501560.2501563}.
%``The Brewing Storm in Cloud Gaming: A Measurement Study on Cloud to End-user Latency'' 

%``Placing Virtual Machines to Optimize Cloud Gaming Experience'' \cite{6853364}  interesting for cloud gaming economics

%Adaptive Mobile Cloud Computing to Enable Rich Mobile Multimedia Applications \cite{6413270}

%Addressing Response Time and Video Quality in Remote Server Based Internet Mobile Gaming \cite{5506572} optimize mobile cloud gaming based on a certain impairment function and by reducing video bitrate and fps (even sub 10fps...)

%``Kahawai: High-Quality Mobile Gaming Using GPU Offload'' \cite{Cuervo:2015:KHM:2742647.2742657}

%``Outatime: Using Speculation to Enable Low-Latency Continuous Interaction for Mobile Cloud Gaming'' \cite{Lee:2015:OUS:2742647.2742656}

%``Assessing the Impact of Game Type, Display Size and Network Delay on Mobile Gaming QoE'' \cite{beyer2014typedisplaydelayimpact} Another user study regarding context factors like screen size and their impact on MOS, but also game type and delay on MOS

%A Method For Feedback Delay Measurement Using a Low-cost Arduino Microcontroller \cite{beyermethod} already covered in measurement methods section

%``QoE Assessment of Interactivity and Fairness in First Person Shooting with Group Synchronization Control'' \cite{Ida:2010:QAI:1944796.1944806} interessant für lag compensation betrachtungen

%``The Impact of Video Encoding Parameters and Game Type on QoE for Cloud Gaming: a Case Study using the Steam Platform'' \cite{slivarimpact}

%``How Do New Visual Immersive Systems Influence Gaming QoE?'' \cite{hupontnew} Vergleich Immersion am monitor vs oculus. beispiel eines schlechten testsetups, da unterschiedliche FoV für beide ausgabetypen (75° vs 100°), test wird stark verfälscht

%%``An experimental estimation of latency sensitivity in multiplayer Quake 3''  vs. \cite{1266180} ``The Effects of Loss and Latency on User Performance in Unreal Tournament 2003'' \cite{Beigbeder:2004:ELL:1016540.1016556}. both papers completely contradict each other: Quake 3: significant impact of latency on game performance (kills/minute); UT3: no impact on user performance at all (kills/deaths per game)

%``Security issues in online games'' \cite{doi:10.1108/02640470210424455}
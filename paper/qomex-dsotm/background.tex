%!TEX root = paper.tex
%%%%%%%%%%%%%%%%%%%%%%%%%%%%%%%%%%%%%%%%%%%%%%%%%%%%%%%%%%%%%%%%%%%%%%%%%%%%%%%%
\section{Background}
\label{sec:background}

Before looking at the fallacies in the literature, first some fundamental gaming concepts need to be introduced. Games are essentially feedback-directed real-time simulators. The game reads player input, updates the game state, and renders new screen contents. This loop repeats as long as the game is running. The \textit{tickrate} governs the frequency of game state updates, and the \textit{framerate} determines the update rate of the output image. Popular examples for the tickrates of games servers include \SI{64}{\hertz} or \SI{128}{\hertz} for \textsc{CS:GO}, \SI{20}{\hertz} for \textsc{Minecraft}, or \SI{30}{\hertz} for \textsc{Dota 2}.

Motion in video data is based on the principle of \textit{apparent motion}, requiring a minimum framerate of about \SI{16.67}{\hertz} for motion perception to work correctly. Traditional video media that play back at fixed framerates, from \SIrange{24}{30}{\hertz}, are considered to be at the low end of motion perception. The sometimes still visible stuttering is often concealed due to camera artifacting such as motion blur. Video games have to target higher framerates: e.g. \SI{30}{\hertz}, \SI{60}{\hertz}, or \SI{120}{\hertz}, depending on the type of game. Higher framerates are especially important for increasing the interactivity and reactivity as video games constantly require input on short time scales to which the game reacts and displays the feedback.

Lag is a crucial factor for almost all games, as it governs the reaction time to in-game events. In literature it is often described solely on the basis of the network delay in an online game, neglecting other components that contribute to the lag, including the input device, the time to sample and process the input, the game engine and server and their tickrates, frame rendering time, and ultimately the time to display the frame on the monitor. Only if all sources are factored in the complete \textit{\gls{E2E} lag} is captured. To properly record the lag some approaches come to mind. Recording in software the output stream of a video game might be the simplest approach to determine video game lag and framerate (e.g., cf. \cite{Chen:2011:MLC:2072298.2071991}). However, this does not represent the complete \gls{E2E} lag however, as both the controller and screen output delay are missing. A 2013 paper \cite{6574660} investigates the quality of cloud gaming interactiveness (i.e., the lag) as well as image quality by employing software recording methods on the client's computer. However, this method assumes a constant delay of game actions and may not capture the actual \gls{E2E} lag of many of real game actions, as they are typically different from and longer as the latency of displaying a menu. A 2013 paper \cite{6574660} investigates the quality of cloud gaming interactiveness (latency) as well as image quality by employing software recording methods on the client's computer. One method to fully capture the \gls{E2E} lag is to simultaneously record both the screen and input device through external measurements. When, e.g., using a video camera the experimenter then counts the frames between pressing a button on the input device and the action appearing on the screen and calculates the lag from this.



%!TEX root = paper.tex
%%%%%%%%%%%%%%%%%%%%%%%%%%%%%%%%%%%%%%%%%%%%%%%%%%%%%%%%%%%%%%%%%%%%%%%%%%%%%%%%
\section{Introduction}
\label{sec:introduction}

The video games industry currently is one of the largest entertainment sectors in terms of revenue, having grossed an estimated \SI{81}[\$]{Bn} in 2014\footnote{\url{http://www.newzoo.com/insights/top-100-countries-represent-99-6-81-5bn-global-games-market/}}. Similarly, it has also garnered some interest from the computer networking and the \gls{QoE} research communities, both taking a look at network-related properties and the resulting subjective quality of individual video games. Many of these research efforts were triggered through the contrivance of cloud gaming. The quality issue of online games still remains a hot topic of investigation and research. A quickly growing number of publications concern themselves with assessing the \gls{QoE} of such games. These assessments are usually conducted through meticulously set up user studies in a lab environment and can produce quite fitting results.

However, the inherent diversity of games and their accompanying game mechanics make it difficult to transfer any of these results from one game to another. In contrast to passively consuming videos, games are highly interactive, making the setup of quality assessment studies that much harder. Using and learning those interactive elements does need to be factored in to such investigations. Compared to plain video streaming, these underlying properties of video games are not that straight-forward to observe from the outside. Yet, in order to conduct proper measurements, it is essential to understand them.

This paper aims to derive lessons learned from gaming user studies conducted in the literature and give insights into some of gaming's core properties. To further this, an additional end-to-end lag model was set up which describes game lag on the basis of other intrinsic game factors, such as the framerate and tickrate. Initial results with this model confirm the influence of the framerate and tickrate on the end-to-end lag. and therefore also on the subjective interaction quality of the game, is much larger than expected. This means that these two parameters need to be tightly controlled in subjective quality assessment studies.




%Many of the past publications regarded the \gls{QoE} of online and cloud video games as ``good enough''. This statement is probably true when looking at a wider audience and looking only at certain games, but does not go very well with the core audience group, that shows real interest in those games.


%With the recent successes of competitive video games, the commercial failure of many cloud gaming services, and the rise of local game streaming offerings it might also be time to critically investigate methods used evaluate those video games. as the scientific view on these matters does not seem to be very aligned with actual mechanics and events.

%The issue of large portions of the scientific literature is that video games are treated as black boxes without taking the inner workings of games into account. For online video games this especially means understanding the main game loop (which is running in a distributed fashion) with its tick rates as well as mechanics and implications surrounding the framerate.

%This holds true for both \gls{QoS} measurements as well as subjective \gls{QoE} user studies, which are often parametrized in rather peculiar ways.



% Looking at those past studies many of them also seem to take very peculiar choices in their selection of measurement parameters, seemingly oblivious to central game properties. For example one paper \cite{claypool2007} examines user actions in \textsc{Quake 3} running at framerates down to \SI{3}{\hertz}, which is considerably below the limit where one can still perceive movement. 

%Casual games requirements are usually more relaxed, e.g., due to the social, temporal, and spatial context factors they are played under (reducing the attention of the player) or inadequate or slow input methods (e.g., touch controls). 
%The second topic covers the game's framerate, its interaction with the monitor refresh rate and their implications.


% The first topic discussed here is the game loop and tick rate of online games. Resulting from this process are some interesting consequences in terms of latency and interactivity for online video games, which usually have very stringent requirements regarding input lag. 

%Derived from these properties are suggestions for three different measurement approaches that are suited to capture the full or parts of the end-to-end lag experienced by players. The explanations and best practices given in this paper are a novel input to the scientific literature and have been derived, e.g., from mechanisms described in the gaming press but also from the first author's personal experiences in online and multiplayer video games.

%!TEX root = paper.tex
%%%%%%%%%%%%%%%%%%%%%%%%%%%%%%%%%%%%%%%%%%%%%%%%%%%%%%%%%%%%%%%%%%%%%%%%%%%%%%%%
\section{Introduction}
\label{sec:introduction}

Recently, video games have garnered research interest from both the computer networking and the \gls{QoE} communities, focusing on network-related properties and the resulting \gls{QoS} and \gls{QoE} of individual video games. 
% Many of these research efforts were triggered through the contrivance of cloud gaming, though the user experience of online games still remains a topic of investigation and research. 
A quickly growing number of publications also concern themselves with assessing the \gls{QoE} of such games. These assessments are usually conducted through user studies, and when set up correctly they can produce meaningful results. However, the inherent diversity of games and their accompanying gameplay mechanics make it difficult to transfer results from one game to another. In contrast to passively consuming videos, games are highly interactive, making the setup of such studies much more difficult. Compared to plain video streaming, underlying properties of video games are also not straight-forward to observe from the outside. Yet, in order to conduct proper measurements, it is essential to understand them. This work aims to critically question past gaming user studies and derive lessons learned, as well as to give insights into some of gaming's core properties. To further this, an \gls{E2E} lag model is introduced which represents the lag between a user input event and the display of the event's results on the screen. The model describes game lag on the basis of other intrinsic game factors, such as the framerate and tickrate. Initial results with this model confirm the influence of framerate and tickrate on the \gls{E2E} lag and therefore also on the game's subjective interaction quality. This means that these two parameters need to be tightly controlled in subjective quality assessment studies.

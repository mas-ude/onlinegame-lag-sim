%!TEX root = paper.tex
%%%%%%%%%%%%%%%%%%%%%%%%%%%%%%%%%%%%%%%%%%%%%%%%%%%%%%%%%%%%%%%%%%%%%%%%%%%%%%%%
\section{Introduction}
\label{sec:introduction}

The video games industry currently is one of the largest entertainment sectors in terms of revenue, having grossed an estimated \SI{81}[\$]{Bn} in 2014\footnote{\url{http://www.newzoo.com/insights/top-100-countries-represent-99-6-81-5bn-global-games-market/}}. Similarly, it has also garnered some interest from the computer networking and the \gls{QoE} research communities, both taking a look at network-related properties and the resulting subjective quality of individual video games. Many of these research efforts were triggered through the contrivance of cloud gaming. The quality issue of online games still remains a hot topic of investigation and research. A quickly growing number of publications concern themselves with assessing the \gls{QoE} of such games. These assessments are usually conducted through meticulously set up user studies in a lab environment and can produce quite fitting results.

However, the inherent diversity of games and their accompanying game mechanics make it difficult to transfer any of these results from one game to another. In contrast to passively consuming videos, games are highly interactive, making the setup of quality assessment studies that much harder. Using and learning those interactive elements does need to be factored in to such investigations. Compared to plain video streaming, these underlying properties of video games are not that straight-forward to observe from the outside. Yet, in order to conduct proper measurements, it is essential to understand them.

This paper aims to derive lessons learned from gaming user studies conducted in the literature and give insights into some of gaming's core properties. To further this, an additional end-to-end lag model was set up which describes game lag on the basis of other intrinsic game factors, such as the framerate and tickrate. Initial results with this model confirm the influence of the framerate and tickrate on the end-to-end lag and therefore also on the subjective interaction quality of the game. This means that these two parameters need to be tightly controlled in subjective quality assessment studies.

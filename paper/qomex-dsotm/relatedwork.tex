%!TEX root = paper.tex
%%%%%%%%%%%%%%%%%%%%%%%%%%%%%%%%%%%%%%%%%%%%%%%%%%%%%%%%%%%%%%%%%%%%%%%%%%%%%%%%
\section{Issues of Past Studies}
\label{sec:relatedwork}

The outcome of past video game \gls{QoE} assessments depended strongly on a wide selection of factors (e.g., on the precise setup, the game, and the choice of players), limiting their validity, comparability, and generality. This is oft reinforced by an apparent lack of gaming-specific knowledge and an improper choice of game and study parametrization, that many papers exhibit. E.g., one paper \cite{claypool2007} examines user actions in \textsc{Quake 3} running at framerates as low as \SI{3}{\hertz}, which is considerably below the threshold of apparent motion, while \cite{5506572} attempts to derive ``gaming quality'' by measuring the \acrshort{PSNR} and tests at framerates between \SI{25}{\hertz} and \SI{6}{\hertz}. Yet, they somehow still note a serviceable quality at those levels.

The precise choice of game and methodology also critically impacts the results, which are almost always non-transferable to any other setting or game, even in the same genre. This circumstance is neglected in most works as they overly generalize their results. 
% e.g. compare the contradicting studies in \cite{1266180}  and \cite{Beigbeder:2004:ELL:1016540.1016556} on Quake 3 and UT3 respectively.
Other publications also have selected entirely questionable metrics to observe gaming \gls{QoE}, e.g. attempting to measure the influence of and tolerance to network delay with a metric that operates on a scale of tens of minutes (as observed in \cite{Claypool:2006:LPA:1167838.1167860}). The challenges such work faces is easy to understand, considering the complexity and variability of video games and the difficulties of finding and setting up good comparable scenarios and testbeds. This makes it much more important to better understand the basic components and underlying objective metrics.


%A second paper \cite{claypool2007} furthers this notion of the influence of network \gls{QoS} on in-game actions and specifically looks at player performance in first person games. A user study records the performance in artificially created in-game environments. Here, the performance gets worse with a degraded network, albeit at a very slow pace.

%Finally, an ITU-T Recommendation \cite{mollertowards} concerning subjectively measuring video game \gls{QoE} is also in preparation, which discusses game-relevant \gls{QoS}-metrics as well as the selection of players and games.

%The ``kills per minute'' of normal players in the \gls{FPS} \textsc{Quake 3} are investigated by \cite{1266180}, which sees a steady decline of this subjective performance metric when increasing the network delay. However, another paper \cite{Beigbeder:2004:ELL:1016540.1016556}, looking at player performance in \textsc{Unreal Tournament 2003} in a controlled in-game environment, finds almost no influence of increased delay and packet loss, even at high values of \SI{200}{\milli\second}. %The discrepancy between the otherwise similar studies could potentially be attributed to the specific construction of the in-game test environment.

%A 2002 paper \cite{Pantel:2002:IDR:507670.507674}, which gathered objective \gls{QoE} measurements from a custom-made RC racing video game, again sees a strong dependence of the player's performance on the delay, and suggests that a network \acrshort{RTT} of \SI{200}{\milli\second} is barely usable and \SI{500}{\milli\second} completely unusable. Finally, the authors of \cite{Bredel:2010:MSR:1944796.1944797} also find a strong and negative influence of high delay on the player's performance, in this case again in \textsc{Quake 3}.

%A 2006 paper \cite{Claypool:2006:LPA:1167838.1167860} categorizes player actions and their relationship to latency, with special regards for the precision and deadlines of actions. However, the in-game metrics under study can not represent short-term effects of latency as they operate on much larger time-scales. (e.g. researching the whole technology tree in Warcraft 3 is neither a representative action of the game nor delay-sensitive at all). Also the presence of lag compensation techniques is not considered. The end result is a delay tolerance table which probably is not very representative of modern video games.

%Adaptive Mobile Cloud Computing to Enable Rich Mobile Multimedia Applications \cite{6413270}

%``Assessing the Impact of Game Type, Display Size and Network Delay on Mobile Gaming QoE'' \cite{beyer2014typedisplaydelayimpact} Another user study regarding context factors like screen size and their impact on MOS, but also game type and delay on MOS

%``QoE Assessment of Interactivity and Fairness in First Person Shooting with Group Synchronization Control'' \cite{Ida:2010:QAI:1944796.1944806} interessant für lag compensation betrachtungen

%``The Impact of Video Encoding Parameters and Game Type on QoE for Cloud Gaming: a Case Study using the Steam Platform'' \cite{slivarimpact}

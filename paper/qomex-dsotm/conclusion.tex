%!TEX root = paper.tex
%%%%%%%%%%%%%%%%%%%%%%%%%%%%%%%%%%%%%%%%%%%%%%%%%%%%%%%%%%%%%%%%%%%%%%%%%%%%%%%%
\section{Conclusion}
\label{sec:conclusion}

The current state of \gls{QoE} research of online and cloud gaming leaves lots of room for development. Issues with past works include inapt metrics, trial conditions that are unrealistic or unrepresentative for games, and disregard of relevant components of lag in games besides network delay (i.e. framerates and tickrates in particular). Our contribution is a simplified \gls{E2E} lag model and simulation which aims to uncover and correct these shortcomings.

%Our end-to-end lag model comprises user input, a command message queue, network delays, the game server tickrate and processing time, video encoding and decoding (for cloud games), and the client's framerate. Examining different values for the parameters, we complement previous work by showing that network \acrfull{QoS} does not fully capture the end-to-end lag by necessity, and why. Indeed, the game server tickrate and the client framerate induce significant amounts of lag, an observation we find undervalued in previous studies.

%The end-to-end lag is a crucial element for the experienced quality of video games. The creation of an abstract model to this lag was the task of this paper. Understanding and modeling the contributing lag factors can build a necessary foundation in order to conduct meaningful subjective user game studies. Additionally, user studies could even be avoided altogether if a good mapping from the end-to-end lag and other basic factors to a \gls{QoE} metric can be derived. 

%Of special interest to the \gls{QoE} research community is probably the masking effect low values of the framerate or tickrate have over the network delay on the end-to-end lag. This and similar side effect need to be considered for future research efforts.

%This work leaves a lot of room for future work, as just the most fundamental model has been introduced here. Much more effort needs to be spent on fleshing out and testing the game classification criteria for example. But once this has been done, such game categories can fulfill a critical purpose for testing video games in conjunction with proper models for basic game behavior, like the end-to-end lag in the example of this publication.

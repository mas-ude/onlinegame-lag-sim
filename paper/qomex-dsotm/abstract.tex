%!TEX root = paper.tex
%%%%%%%%%%%%%%%%%%%%%%%%%%%%%%%%%%%%%%%%%%%%%%%%%%%%%%%%%%%%%%%%%%%%%%%%%%%%%%%%
\begin{abstract}

Due to the nature of video games and their large diversity, results of past assessments of video game \acrshort{QoE} have often been limited to one single game, and were generally difficult to transfer to any other game, even if they seem to be similar on the surface. In addition, studies often did not properly valuate certain game properties, such as the framerate, impacting their credibility.

With the help of examples of past literature we discuss the importance of the framerate and its impact on user studies. Furthermore, our simulation model explains the importance of such properties and their handling in objective measures by putting the \gls{E2E} lag in relation to the framerate.

%This work aims to investigate one of the key components of gaming \acrshort{QoE}: The end-to-end lag. It has previously only been investigated in the context of individual games and lacks the basic methodology for a game-independent examination. The key contribution of this paper is a model for the end-to-end lag that is based on its individual factors, including the network delay, the framerate, and the game's tickrate.

\end{abstract}

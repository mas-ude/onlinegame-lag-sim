%!TEX root = paper.tex
%%%%%%%%%%%%%%%%%%%%%%%%%%%%%%%%%%%%%%%%%%%%%%%%%%%%%%%%%%%%%%%%%%%%%%%%%%%%%%%%
\section{Introduction}
\label{sec:introduction}

% A frequent motif in online video game \gls{QoE} assessments (for both multiplayer and cloud games) is the apparent obliviousness of such studies to the inner workings of video games.

A growing number of publications concerns itself with assessing the subjective quality of online video games (in both multiplayer and cloud games). These assessments are usually conducted through meticulously set up user studies in a lab environment and can produce quite fitting results. Yet, the inherent diversity of games and their accompanying gameplay mechanics makes it difficult to transfer any of these results from one game to another. Games are an interactive medium and are not passively consumed as videos are, greatly increasing the complexity of video game quality study setups. Gaming is governed by much more intricate and engaging properties, that need to be factored in for such investigations. Compared to plain video streaming, these underlying implementation properties of video games are not that straight-forward to observe from the outside.

This paper aims to give insights to one of the bespoken core properties, namely the ``\textit{\gls{E2E}}'', and its influence in online and cloud gaming scenarios. This lag is a main governing factor in determining the interaction quality of video games as a higher lag means an apparent disjoint between a player's inputs and the resulting game's visible reactions. To this end, this paper, as a continuation of previous work conducted in \cite{metzger2016gamesframes}, describes a general model of this lag on the basis of intrinsic game and interaction factors, especially the framerate and tickrate. To demonstrate the model, this paper further provides a simulation implementing typical gaming scenarios. Results from this simulation indicate that the influence of both framerate and tickrate on the end-to-end lag, and therefore also on the subjective interaction quality of the game, is much larger than expected. This means that these two parameters need to be tightly controlled in subjective quality assessment studies.

~\newline
This paper is structured as follows. First, in Section~\ref{sec:background} gives an introduction to important parts of video game engine terminology and properties, providing a big-picture for game properties relevant to this model. Afterwards, Section~\ref{sec:model}, describes the abstract model to describe \gls{E2E} lag, followed by a simulation parameter study in order to identify main influence factors for the \gls{E2E} lag in Section~\ref{sec:simulation}. The paper concludes in Section~\ref{sec:conclusion} with a discussion of key findings and providing pointers to future work.


% TODO: touch the QoE perspective as we're submitting to PQS, i.e. perceptual quality focus
% TODO-reviewer: Which cases does the proposed queueing model cover and which does it not?
% TODO-reviewer: How was the model verified? Are the authors comparing their model against a game? It could be an open-source game where the mechanics are known.
% TODO-reviewer: incorporate \cite{Ivkovic:2015:QMN:2702123.2702432}
% TODO-reviewer: The first sentence in the abstract seems to be broken

%!TEX root = paper.tex
%%%%%%%%%%%%%%%%%%%%%%%%%%%%%%%%%%%%%%%%%%%%%%%%%%%%%%%%%%%%%%%%%%%%%%%%%%%%%%%%
\section{Introduction}
\label{sec:introduction}

A common trend in online video game \gls{QoE} assessments (for both multiplayer and cloud games) is the apparent obliviousness of such studies to the inner workings of video games. This especially means understanding the main game loop with its tick rates as well as mechanics and implications surrounding the framerate. The diversity of games and their accompanying gameplay mechanics also makes it difficult to transfer any findings from one game to another. Yet, in order to conduct proper measurements, it is essential to understand these mechanics, resulting in a better, quantitative classification of games based on game \textit{properties} rather than opaque and less effective \textit{categories} like \gls{FPS} or \gls{RPG}, and thus a basis for an objective quality assessment model. To facilitate this, this work investigates gaming properties, especially a game's framerate and tickrate, on the basis of the ``\gls{E2E} lag'' using a model and queuing simulation. Initial results visualize that the influence of the framerate and tickrate on the \gls{E2E} lag, and therefore also on the subjective quality of the game, is much larger than expected. This means that these two parameters need to be tightly controlled in subjective quality assessment studies.



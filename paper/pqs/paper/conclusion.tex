%!TEX root = paper.tex
%%%%%%%%%%%%%%%%%%%%%%%%%%%%%%%%%%%%%%%%%%%%%%%%%%%%%%%%%%%%%%%%%%%%%%%%%%%%%%%%
\section{Conclusion}
\label{sec:conclusion}

A proper setup of gaming \gls{QoS}/\gls{QoE} studies is of critical importance to their validity. The \gls{E2E} lag queuing model set up in this work can support these endeavors a long way through an improved understanding of relevant game properties and their interactions. The role of the framerate, and its lag-inducing effects has been undervalued in the past, which these models and simulations aim to rectify. Of special interest is the masking effect low values of the framerate or tickrate have on the network delay on the \gls{E2E} lag. This and similar side effect need to be considered for future research efforts.


This work leaves much room for future work, as just the most fundamental model has been introduced here. Further effort needs to be spent on fleshing out game classification criteria for example. But once this has been done, such game categories can fulfill a critical purpose for testing the user experience of video games in conjunction with proper models for basic game properties, such as the investigated framerate and lag.

%!TEX root = paper.tex
%%%%%%%%%%%%%%%%%%%%%%%%%%%%%%%%%%%%%%%%%%%%%%%%%%%%%%%%%%%%%%%%%%%%%%%%%%%%%%%%
\begin{abstract}
We present a queueing model and first \gls{DES} results for \gls{E2E} 
lag in networked and cloud computer games. \gls{E2E} lag describes the 
latency between a user's action and the display of the action's results 
on the screen, thus providing a \gls{QoE} metric for interactivity.
In contrast to models that focus on the network \gls{RTT}, we also 
factor in the game's tickrate and framerate (and codec latencies in the 
case of cloud games).
% The model's formalism allows us to deduce a few fundamental 
% relationships between the parameters.
We use the model to explore the parameter space, and explain how 
desynchronization effects between the server tickrate and the game 
framerate severely impact the \gls{E2E} lag, even up to the point 
that network delay is entirely masked by other lag contributions. 
This helps to put into perspective existing literature that uses 
questionably low framerates and only focuses on network parameters 
to study game \gls{QoE}.

\end{abstract}
\noindent{\bf Index Terms}: gaming, end-to-end lag, queuing model
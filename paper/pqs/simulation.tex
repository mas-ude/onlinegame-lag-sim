%!TEX root = paper.tex
%%%%%%%%%%%%%%%%%%%%%%%%%%%%%%%%%%%%%%%%%%%%%%%%%%%%%%%%%%%%%%%%%%%%%%%%%%%%%%%%
\section{Simulation and Scenario Evaluation}
\label{sec:simulation}

Based on the models introduced in Section~\ref{sec:model} a stochastic \gls{DES} was created. This \textsc{Gnu R}-based simulator strives to translate all of the earlier discussed components of lag, while keeping them as configurable as possible.
%By using a vector-based language the core loop is kept compact and easy to modify in order to adapt to new scenarios in addition to the ones evaluated here. 
Due to the influence of several stochastic processes (user inputs $U$, network delay $D$, server processing time $P$) and the differing offset of the clocked processes, a sufficient number of repetitions is required to provide meaningful results. This is achieved by running large numbers of simulations in parallel through the built-in R facilities.

This section presents simulation results on each of the flavors of how to play video games: the local, online, and the cloud scenario. The investigations here are not intended to cover every aspect of all parameters or to conduct full parameter studies, but are rather to highlight some particular aspect in each of the scenario.

All of the source code and the data from the upcoming scenarios is provided in the simulation folder\footnote{\url{https://github.com/mas-ude/onlinegame-lag-sim/tree/master/simulation}} of the repository and can be used as public domain.


\subsection{Parameters}

Besides being fully reconfigurable, the simulation currently also has a few default parameters set and makes assumptions that go beyond what the base end-to-end model implies for reasons of simplicity. They are chosen to be as realistic as possible and and are summarized in Table~\ref{tab:notation}.
%\hoss{Vielleicht noch kurz erwaehnen, dass die Parameter realistisch gewaehlt sind? Gibt es dafuer Referenzen?}
% fm: referenzen für solche werte sind leider eher schwer zu finden

First, the input is modeled by an exponential distribution with a rate of $20$ events per second. The offsets between the clocked processes are uniformly distributed in their respective intervals. It is further assumed that the server processing time $P$ follows a left-truncated normal distribution with $\mu_P = \SI{3}{\milli\second}$ and $\sigma_P = \SI{0.1}{\milli\second}$.
%\textbf{TODO:} implications and reasons?
Additionally, the rate $c$ at which command messages are sent to the game server is set equal to the server's tickrate $g$ as the server would not process more commands either way. This might however increase the end-to-end lag in some situations if a command message just misses one of the server's ticks and has to wait another full cycle.

The evaluation of the presented scenarios was conducted on the basis of $r=\np{1000}$ repetitions for each setting. Each run $i \in \{1,\dots,r\}$ consisted of a series of $n=\np{100}$ input events and their associated end-to-end lag $T_{i,j}$ for the $j$-th input event. On this basis a median lag $\widetilde{T_i}$ was calculated over the $n$ input events for each run $i$, and finally an overall mean lag out of the individiual medians of each repetition, i.e. $\overline{T}=\frac{1}{n}\sum_{i=1}^r\widetilde{T_i}$.


%%%%%%%%%%%%%%%%%%%%%%%%%%%%%%%%%%%%%%%%%%%%%%%%%%%%%%%%%%%%%%%%%%%%%%%%%%%%%%%%
\subsection{Local Games}

The first and simplest scenario to be investigated here is the case of the local game. In the version implemented here, the tickrate is still present, but the influence of the network is entirely removed. It therefore represents the best case an online multiplayer game could achieve if the server ran locally. Figure~\ref{fig:nwless-scatter} plots the relationship between the frame duration (i.e., the inverse of the framerate), the tick duration, and the resulting median end-to-end lag.
% NOTA BENE: Every status update that the server sends must wait for another frame time until we consider it rendered.
%
% The scatter plot of raw e2e lag values for one (FR, TR) configuration thus spans
% * separate "boxes" for each CR/FR/TR offset config
% * inside of which the input events U scatter
% * Boxes are, when all other rates/processes are in optimal sync,
%   - bound from below by the frame-time f^-1 as the minimum
%   - bound from above by f^-1 + c^-1, because c "groups together" input events
% * alternatively, when maximally *out of* sync,
%   - the boundaries shift upwards by another f^-1 + g^-1, resulting in a total just short of c^-1 + g^-1 + 2*f^-1: Input event waits almost-full command cycle, which then waits almost-full tick, which then waits almost-full frame; plus one frame per the "nota bene" above.
% * a "cloud of boxes" (per the above description) whose minima are uniformly distributed (from visual inspection).
%
% The distributions of values for different (FR, TR) configs differ in their characteristics pretty starkly:
% * for low framerates and high tickrates, we get an approximately uniform lag distribution. Seems that the FR which is "further down the tandem queue" dominates the lag, so its uniformly distributed phase offset governs the lag distribution.
% * for high framerates and low tickrates, it approaches a truncated normal distribution.
The values can be estimated as follows. Every user input event traverses three queues with fixed-rate outputs ($c=g$, g, and $f$), and every game state update waits for the next full frame until it gets displayed. In the ideal case, an input event occurs just before the command message is sent off, the server tick follows as soon as the message is received, and the update reaches the client just before a frame is output. Then the end-to-end lag is slightly above the frame time that must elapse before the update can be rendered to screen, $T_{min}>f^{-1}$.

In case the events are unfavorably offset against one another, an input event has to wait almost a full command-message cycle until it is sent on; it reaches the server just after a tick has occurred, so it waits almost a full server tick; finally, it will wait for almost two frame times until it is displayed. Combining with the previous result bounds the lag as follows, $f^{-1} < T < c^{-1}+g^{-1}+2f^{-1}$. The mid-interval point between these two limits is $T_{mid}=\frac{3}{2} f^{-1} + g^{-1}|_{c=g}$ which coincides roughly with the medians of the stochastic simulation.


Looking at a typical \SI{60}{\hertz} video game with an equal tickrate (i.e. a frame duration of $\approx \SI{16.6}{\milli\second}$) the median lag is in the range of \SIrange{45}{50}{\milli\second}. So even under quasi-optimal circumstances, there is already a considerable amount of end-to-end lag (even without factoring in the delay of the screen and input devices) that can only increase with the presence of network delay. Therefore, video games (and quality assessments thereof) should try to achieve the highest framerate possible to minimize this influence.

%%%%%%%%%%%%%%%%%%%%%%%%%%%%%%%%%%%%%%%%%%%%%%%%%%%%%%%%%%%%%%%%%%%%%%%%%%%%%%%%
\subsection{Online Gaming}

Next up is the full scenario of an online video game, now with network lag $D$ and server processing time $P$ included. For this exemplary scenario, the one-way delay $D$ was assumed to follow a left-truncated normal distribution, with $\mu_D = \SI{20}{\milli\second}$ and $\sigma_D = \SI{5}{\milli\second}$, producing non-negative network delays. Typical competitive online games today are expected to operate in such ranges. An \acrshort{RTT} of \SI{100}{\milli\second} is often considered to be the upper limit for a good playing experience. $P$ is similarly left-truncated. In addition to the components in the previous scenario, the lag now also contains contributions by the network round-trip and processing delay, $2D + P$.

\begin{figure}[!t]
	% TODO: adjust the bounding box of the 3dbar plot, instead of playing with fire (aka \vspace)
	\centering
	\vspace{-6mm}
	\includegraphics[width=1.0\columnwidth]{../../simulation/visualization/e2e-lag-3dbars.pdf}
	\vspace{-15mm}
	\caption{Influence of client framerate and server tickrate on the median end-to-end lag in the online game scenario. For high rates $f$, $g$, the lag approaches \SI{43}{\milli\second}.}
	% TODO: \hoss{Kann man hier einen stacked bar plot bauen, bei dem man den Networking Anteil sieht? Das wuerde die Aussage gut unterstuetzen.}
	% TODO: nicht rechtzeitig für submission deadline, aber für eine nächste version
\label{fig:3dbars-framerate-tickrate-lag}
\end{figure}

Figure~\ref{fig:3dbars-framerate-tickrate-lag} shows a 3D bar plot of the influence of both the framerate and the tickrate on this scenario. The axes mark typical values for $f$ and $g$. Two things can be noted here. First, the framerate has a larger influence on the lag than the tickrate. Second, for low framerates and tickrates, the impact of network delay on the end-to-end lag is almost completely masked. Only if both rates are high enough, the network delay will play a more significant role. This masking effect has large implications for video games and their evaluation. Many evaluations examine the influence of the network delay only, without considering other contributions to end-to-end lag. Our results indicate that this might not be the best course of action. The effect likely shifts to lower values of the frame- and tickrates when a higher network delay is examined.

Another interesting result (not plotted here) is the much larger variance of lag in the framerate dimension when compared to the tickrate. This requires video game studies to have a very high repetition rate to provide meaningful results.


%%%%%%%%%%%%%%%%%%%%%%%%%%%%%%%%%%%%%%%%%%%%%%%%%%%%%%%%%%%%%%%%%%%%%%%%%%%%%%%%
\subsection{Cloud Gaming}

Finally, we construct a Cloud Gaming scenario. The tickrate has been removed;  instead, a constant encode ($e$) and decode ($d$) delay is in place at the game streaming server and client respectively. The frames are now rendered by the server, so they need to be transported back to the client first. Instead of assuming a specific network throughput and frame size, we simply add one frame time to account for the transmission of the encoded screen contents.
%Without knowing the connection's throughput and absolute values for the frame sizes, the simulation simply applies an upper limit for the transmission duration, namely once again the frame's duration, or the inverse of the framerate. 
For the network delay $D$, the same values as for the online game are used. $c$ is set to $\SI{200}{\hertz}$.

\begin{figure}[!t]
	\centering
	%\includegraphics[width=1.0\columnwidth]{images/e2e-delay-sim.pdf}
	\includegraphics[width=1.0\columnwidth]{../../simulation/visualization/cloudgaming-lag-cdf.pdf}
	\caption{\acrshort{ECDF} of the influence of the rendering and streaming framerate on the end-to-end lag in the cloud scenario. Vertical intercept denotes the average base delay of \SI{68}{\milli\second}.}
	%\hoss{On average, a network delay of $\unit[2\cdot20=40]{ms}$ (correct???) exists. A constant encoding time (\unit[15]{ms}) and decoding time (\unit[5]{ms}) is assumed. Thus, the major part of the e2e lag is caused by the framerate setting. Kann man hier noch eine Linie reinzeichnen, die die Summe 5+15+40=60 ms zeigt?}}
	%
	% Albert: Uh, oh, the cloud simulation seems to have reused a stray 
	%         200 Hz setting for $c$ from a previous online game sim.
	%         I'm documenting that this value has been used.
	%
\label{fig:cloud-e2e-delay-sim}
\end{figure}

Figure~\ref{fig:cloud-e2e-delay-sim} shows the results of this scenario as an \gls{ECDF} of the end-to-end lag for several framerates. As before, the framerate impacts the end-to-end lag more severely than the network delay.
%the large influence of the framerate when compared to the network delay is evident. 
This result is of particular importance considering how past studies have relied on similarly low framerates as $5-\SI{15}{\hertz}$ when assessing the network influence on cloud gaming. Similarly, these results can provide guidelines for implementors of cloud gaming to factor in the framerate in their calculations accordingly.



%%%%%%%%%%%%%%%%%%%%%%%%%%%%%%%%%%%%%%%%%%%%%%%%%%%%%%%%%%%%%%%%%%%%%%%%%%%%%%%%
\subsection{Representative Games for QoE studies}
\label{sec:game-criteria}
%\hoss{Ich glaube der Titel triffts noch nicht so ganz. Choosing Representative Games for QoE Studies? }
The three scenarios presented here serve to provide initial insights into the complex interactions of the end-to-end lag. Although the underlying abstract model adopts some simplifications and some properties are not factored in yet, the results are still very revealing. Using the model and simulator as baseline, one can get a good estimation of the expected video game \gls{QoS} values. Alternatively, it can help in choosing representative games for select scenarios.


The result of any game quality assessment strongly depends on the considered game, as each game is a unique collection of individual game mechanics, each bringing a distinct influence onto the assessment alongside with it. Here we provide a foundation for quality metrics by rather investigating game-independent metrics, including the framerate, frame \gls{IAT}, and the end-to-end lag.


%%%%%%%%%%%%%%%%%%%%%%%%%%%%%%%%%%%%%%%%%%%%%%%%%%%%%%%%%%%%%%%%%%%%%%%%%%%%%%%%
\subsubsection{Game Independent Metrics}
%\hoss{Was soll hier gesagt werden? Selecting the right parameter range for framerate?}
Framerate and frame \gls{IAT} are a good indicators for the game's fluidity and responsiveness to inputs. As such, choosing the right range for these parameters is especially important in cloud gaming scenarios or for quality assessments. End-to-end lag gives the best overall picture on the attainable gaming experience and should always be preferred over partial lag values, e.g. by purely investigating the network latency. The impact of the lag also depends on the type and precision of controls the game offers. For example, a keyboard and mouse driven PC game might be much more sensitive to high lag values than a mobile game with touch controls.

Input controls are but one aspect of the games environment and settings, which need to be carefully selected to achieve meaningful results. This especially concerns PC games which usually offer a wide range of options to choose from. Here, the graphics options are the most impactful. The recommended settings to run games at are a video resolution of 1080p or higher with the games other graphics options set to high or at least medium values in order to reflect a typical gaming experience. Due to the demonstrated impact of the framerate on the end-to-end lag a target of \SI{60}{\hertz} should considered as a minimum rate for most games. Some types of games are less dependent on the framerate, where a rate of \SI{30}{\hertz} would still be considered acceptable.

% Experimenters should never set a framerate lower than that for the reasons discussed in the previous section. They should however also consider testing at higher framerates, especially for competitive games with a high tickrate to further reduce the negative impact of low framerates.

%The general focus here lies on measuring online games with high demands. The idea is that if these games work at an objectively good quality, it is reasonable to assume that all other games do so as well.


%%%%%%%%%%%%%%%%%%%%%%%%%%%%%%%%%%%%%%%%%%%%%%%%%%%%%%%%%%%%%%%%%%%%%%%%%%%%%%%%
\subsubsection{Quantifiable Game Classification Criteria}

Finding games which are representative for certain input and lag demands is challenging. For example, the traditional game genre categorization is not a good starting point, as games from the same category can be vastly different in terms of game speed and necessary reaction times. Rather, the following four exemplary metrics might prove useful when classifying games to better assess the impact of the end-to-end lag on the experienced quality.

\textbf{Required number of decisions or actions in a certain time span} E.g., the \gls{CCG} \textsc{Hearthstone} may only require a handful of actions, i.e. choosing and playing cards, each turn, while in order to competitively play the \gls{RTS} \textsc{Starcraft 2} you more or less need to achieve a few hundred \gls{APM}, with the record being higher than $800$ \gls{APM}. \cite{6404025} defines a related metric dubbed \textit{command heaviness} comparing the amount of change to the input rate, under the umbrella term \textit{real-time strictness}. Micromanagement-intense games usually tend to result in high APM rates.

\textbf{Maximum successful reaction time to in-game actions} Again e.g., \textsc{Hearthstone} as a turn-based game requires no instant reaction time at all, as the opponent's and the player's actions are separated into turns. First-person shooters like \textsc{Counter-Strike: Global Offensive} are usually on the opposite extreme of this spectrum, as they tend to have a very high tickrate and literally often require you to ``shoot first'' to win. This is also investigated in \cite{Claypool:2006:LPA:1167838.1167860}.

\textbf{Ratio of unpredictable actions} This is closely related to the previous metric. When considering an entirely rhythm-based game (e.g., \textsc{Guitar Hero}), where you can completely pre-plan all of your actions, to a twitch-based shooter like \textsc{Counter-Strike} where you have to react from moment to moment. In theory, a game with no surprising events will be less influenced by a higher end-to-end lag.

\textbf{Temporal and spatial accuracy and precision of input events} Accuracy can be relevant in both temporal as well as spatial aspects. Thus, it can be influenced by both the image quality and the frame rate. For example, discrete events, e.g., button-presses on a controller, require less spatial precision than analogue inputs, e.g., free-form mouse movement.

This is a non-exhaustive list of selected quantifiable criteria. It should be refined in future studies and extended before such metrics can be applied as a means of categorization. Moreover, the value of some of these metrics might not be that easy to determine as they involve playing the game. Even as a qualitative discriminator or when considering broad value ranges, such a classification might be much more sensible than one purely based on the video game genre.

% Ultimately, to capture any and all latency sources in gaming you would need to rely on external recording gear.
% With modified input: zero latency and visible input detection (e.g. solder some LEDs to the buttons)

% Also Arduino with photodiode method described in \cite{beyermethod}
% Both this and camera method also work for closed game consoles


%%%%%%%%%%%%%%%%%%%%%%%%%%%%%%%%%%%%%%%%%%%%%%%%%%%%%%%%%%%%%%%%%%%%%%%%%%%%%%%%
%\subsection{Evaluated Metrics}

%%%%
%\subsubsection{Frame Rates and Frame Times (i.e. frame IAT)}
%i.e. frame IAT
%Reasoning for frame IAT and the negligence of past investigations
%%%%
%\subsubsection{Total and additional end-to-end latency}
%physical controller input to in-game reaction
%different in-game actions have already difference in latency, therefore need to test various actions for a complete picture
%Also discuss RTT as Hz (1/RTT) as measure for interactivity


%%%%%%%%%%%%%%%%%%%%%%%%%%%%%%%%%%%%%%%%%%%%%%%%%%%%%%%%%%%%%%%%%%%%%%%%%%%%%%%%
%\subsection{Reasonable Configuration/Setting Ranges to Test}

% Resolution: Minimum 720p, 1080p recommended, even higher is better (1440p or 2160p)
% Frame rate: 60 fps very much recommended, 30 absolute minimum,  120 or 144 can also be feasible
% Configure games to run at high or at least medium settings
% For console games: use the games intended settings for the console, never downscale the game or reduce the frame rate for streaming
% Assume no network latency higher than 200ms, preferably less than 100ms
% Assume typical access link conditions, i.e. no less than 10-16Mb/s



%Works only for general purpose computing devices with full access.
%Easiest method, but might not capture full end-to-end latency.
%FRAPS, OBS, DirectX Hooking, MSI Afterburner
%FCAT as hybrid solution with external capture card and computer


% \url{http://www.red.com/learn/red-101/high-frame-rate-video}


% articles:
% why frametimes
%     \url{https://techreport.com/review/21516/inside-the-second-a-new-look-at-game-benchmarking}

% Inside the second with Nvidia's frame capture tools
%     \url{https://techreport.com/review/24553/inside-the-second-with-nvidia-frame-capture-tools}

 % As the second turns: the web digests our game testing methods
 %    \url{https://techreport.com/blog/24133/as-the-second-turns-the-web-digests-our-game-testing-methods}

% GPU Reviews: Why Frame Time Analysis is important
%     \url{http://www.vortez.net/articles_pages/frame_time_analysis.html}

% Durante's Witcher 3 analysis: the alchemy of smoothness
%     \url{http://www.pcgamer.com//durantes-witcher-3-analysis-the-alchemy-of-smoothness/}


% Analysing Stutter – Mining More from Percentiles
%     \url{https://developer.nvidia.com/content/analysing-stutter-%E2%80%93-mining-more-percentiles-0}

% fraps vs fcat method
%     \url{http://www.extremetech.com/gaming/154089-after-almost-20-years-gpu-benchmarking-is-moving-past-frames-per-second}

% FRAPS + FRAFS
% \url{http://www.fraps.com/}
% \url{http://sourceforge.net/projects/frafsbenchview/}
% \url{http://www.5group.com/wordpress/2012/07/14/gpu-mist-pre-release-1-0-rc1/}

% issue: Fraps measures the flip queue input rather then the actual render output frames which is fine when measuring FPS but is rather poor if you want to measures actual frame times and analyze microstutter.


% NVIDIA FCAT
% \url{http://www.geforce.com/hardware/technology/fcat}
% \url{http://www.overclockers.com/nvidias-fcat-gpu-testing-pursuing/}


% Valve for Linux GL Games
% \url{https://github.com/ValveSoftware/voglperf}

% Info über MSI Afterburner overlay? oder GF experience? GPU-Z? Rivatuner Statistics Server?
% \url{http://www.overclock.net/a/how-to-use-rivatuner-afterburner-on-screen-display-and-more}



% \url{https://en.wikipedia.org/wiki/Game_classification}
% \url{https://en.wikipedia.org/wiki/Video_game_genre}
% \url{https://en.wikipedia.org/wiki/List_of_video_game_genres}


%!TEX root = paper.tex
%%%%%%%%%%%%%%%%%%%%%%%%%%%%%%%%%%%%%%%%%%%%%%%%%%%%%%%%%%%%%%%%%%%%%%%%%%%%%%%%
\section{Conclusion}
\label{sec:conclusion}

%These scenarios reveal the necessity of a tight control over game parameters, such as the framerate, resolution, or input devices, in accordance with the game's type.

This paper presents a model for \acrfull{E2E} lag in video games, including online and cloud variants. The \gls{E2E} lag represents the time elapsed between a player input event such as mouse movement or keystrokes and the display of the event's results in the game on the local display. This lag is a main governing factor for \acrfull{QoE} in human-computer interaction in general, and video games in particular. The model is parameterized on the command rate at which batched user events are forwarded, the server tickrate and state processing time, the game's local framerate, the network delay (for online games), and codec delays (for cloud games). % Note: We don't look at local latencies really: Mice and keyboards, USB, TV sets, multibuffering, ...

The model is simulated using \acrfull{DES}, showing the dominant influence of the game framerate on the \gls{E2E} lag particularly for low framerates. It may even mask the influence of network delay, yet it appears underrepresented in previous work.  On an abstracted level, the model helps to explain the mechanics behind lag in different game types and architectures. This is of interest to both actual implementations of games and study design for game \gls{QoE} assessment.

Going forward, the model could be included in larger \gls{QoE} frameworks; also, an analytical approach may provide further structural insights, and other lag sources such as input and output devices could extend the model. Lastly, the model should be validated in a practical setting.\\

\textit{Note: To foster participation and independent replication, the model simulation code is available as free, open-source software from the authors' repository~\cite{onlinegame-lag-sim-repo}, as are the raw data.}

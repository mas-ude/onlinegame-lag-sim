%!TEX root = paper.tex
%%%%%%%%%%%%%%%%%%%%%%%%%%%%%%%%%%%%%%%%%%%%%%%%%%%%%%%%%%%%%%%%%%%%%%%%%%%%%%%%
\section{Conclusion}
\label{sec:conclusion}

These scenarios reveal the necessity of a tight control over game parameters, such as the framerate, resolution, or input devices, in accordance with the game's type.


The three scenarios presented here serve to provide initial insights into the complex interactions of the end-to-end lag. Although the underlying abstract model adopts some simplifications and some properties are not factored in yet, the results are still very revealing. Using the model and simulator as baseline, one can get a good estimation of the expected video game \gls{QoS} values. Alternatively, it can help in choosing representative games for select scenarios.


A proper setup of gaming \gls{QoS}/\gls{QoE} studies is of critical importance to their validity. The \gls{E2E} lag queuing model set up in this work can support these endeavors a long way through an improved understanding of relevant game properties and their interactions. The role of the framerate, and its lag-inducing effects has been undervalued in the past, which these models and simulations aim to rectify. Of special interest is the masking effect low values of the framerate or tickrate have on the network delay on the \gls{E2E} lag. This and similar side effect need to be considered for future research efforts.


This work leaves much room for future work, as just the most fundamental model has been introduced here. Further effort needs to be spent on fleshing out game classification criteria for example. But once this has been done, such game categories can fulfill a critical purpose for testing the user experience of video games in conjunction with proper models for basic game properties, such as the investigated framerate and lag.

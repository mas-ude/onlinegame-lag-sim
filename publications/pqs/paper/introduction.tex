%!TEX root = paper.tex
%%%%%%%%%%%%%%%%%%%%%%%%%%%%%%%%%%%%%%%%%%%%%%%%%%%%%%%%%%%%%%%%%%%%%%%%%%%%%%%%
\section{Introduction}
\label{sec:introduction}

\acrfull{E2E} lag, i.e. the delay between an input event and the feedback of the event's result, ubiquitously impacts human interactions with computers. This notion has a rich history, comprising studies on both objective (e.g. task completion times) and subjective (user experience / \gls{QoE}) metrics for various types of interactions. Online multiplayer and cloud video games form the perhaps most recent objects of investigation: Lag is a main governing factor in determining the interaction quality of video games, as a higher lag means an apparent detachment of a player's inputs from the game's resulting visible reactions. This has not been lost on researchers, prompting research on gaming \gls{QoE} given various additional input parameters such as game categories, game task classifications, player activity, etc. %Sources of lag considered include the network, player input devices, monitors/screens/TVs, VSYNC/double/triple buffering

However, online video game \gls{QoE} assessments appear oblivious to the inner workings of video games. This especially means understanding the main game loop with its tickrates as well as mechanics and implications surrounding the framerate.
% The diversity of games and their accompanying gameplay mechanics also makes it difficult to transfer any findings from one game to another. Yet, in order to conduct proper measurements, it is essential to understand these mechanics, resulting in a better, quantitative classification of games based on game \textit{properties} rather than opaque and less effective \textit{categories} like \gls{FPS} or \gls{RPG}, and thus a basis for an objective quality assessment model. 
This paper, as a continuation of previous work conducted in \cite{metzger2016gamesframes}, aims to illustrate the inner workings of the main game loop and highlight the different contributors to \gls{E2E} lag. It describes a general model of \gls{E2E} lag on based on intrinsic game and interaction factors, especially the framerate and tickrate. To demonstrate the model, this paper further provides a simulation implementing typical gaming scenarios. Results from this simulation indicate that the contributions of both framerate and tickrate to the \gls{E2E} lag, and therefore on the subjective interaction quality of the game, easily exceed the influences of, e.g., the network connecting the game client and server.

The results presented and the custom tools that are available as free-open source software from the authors' public repository~\cite{onlinegame-lag-sim-repo}, may assist the design of future subjective quality assessment studies. They point at relevant parameters, and allow for surveying the amount of influence of these parameters.

%Games are an interactive medium and are not passively consumed as videos are, greatly increasing the complexity of video game quality study setups. Gaming is governed by much more intricate and engaging properties, that need to be factored in for such investigations. Compared to plain video streaming, these underlying implementation properties of video games are not that straight-forward to observe from the outside.

~\newline
This paper is structured as follows. First, §\ref{sec:background} gives an introduction to parts of video game engine terminology and properties relevant to this model, %, providing a big-picture for game properties . 
and reviews related work.
Afterwards, §\ref{sec:model} describes the abstract model to describe \gls{E2E} lag, followed by a simulation parameter study in order to identify main influence factors for the \gls{E2E} lag in §\ref{sec:simulation}. The paper concludes in §\ref{sec:conclusion} with a discussion of key findings and future perspectives.




% TODO: touch the QoE perspective as we're submitting to PQS, i.e. perceptual quality focus
% TODO-reviewer: Which cases does the proposed queueing model cover and which does it not?
% TODO-reviewer: How was the model verified? Are the authors comparing their model against a game? It could be an open-source game where the mechanics are known.
% TODO-reviewer: incorporate \cite{Ivkovic:2015:QMN:2702123.2702432}
% TODO-reviewer: The first sentence in the abstract seems to be broken

%!TEX root = paper.tex
%%%%%%%%%%%%%%%%%%%%%%%%%%%%%%%%%%%%%%%%%%%%%%%%%%%%%%%%%%%%%%%%%%%%%%%%%%%%%%%%
\begin{abstract}

Recently, online and cloud video games have been shifting into the focus of consumer and academic interest. Specifically, the \acrshort{QoE} of video games is such an important topic to several parties that it merits a close investigation.
However, due to the nature of video games and their large diversity, the assessment of video game \acrshort{QoE} is always specific to one single game, and arguably even specific to the exact experimental conditions, and are difficult to transfer to any other game, even if they seem to be similar on the surface.

This paper aims to investigate one of the key components to the \acrshort{QoE}: The end-to-end lag of video games. It has previously only been investigated in the context of individual games and lacks the basic methodology for a game-independent examination. The key contribution of this paper is a model for the end-to-end lag that is based on its individual factors, including the network delay, the framerate, and the game's tickrate.
%  Additionally, a parametrizable simulation that calculates the lag for scenarios such as online and cloud gaming is provided. 
The model, and the accompanying simulation of online and cloud gaming scenarios, complements prior work by revealing a remarkable influence of the network-independent factors on the lag.

\end{abstract}
